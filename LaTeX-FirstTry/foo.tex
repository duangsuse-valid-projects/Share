\documentclass[11pt]{article}
\usepackage{amsmath}

% XeTex is utf8 based
%\usepackage[utf8]{inputenc}

\usepackage[T1]{fontenc}
\usepackage{lmodern}

% CJK Support
\usepackage{fontspec, xunicode, xltxtra}

% CJK style
\usepackage[top = 1in, bottom = 1in, left = 1.25in, right = 1.25in]{geometry}

\XeTeXlinebreaklocale "zh"
\XeTeXlinebreakskip = 0pt plus 1pt minus 0.1pt

\renewcommand{\baselinestretch}{1.25}

\parindent 2em

% Set main font
\setmainfont{DejaVu Sans ExtraLight}

\title{$ \LaTeX $ Document Test Page}
\author{duangsuse}

\begin{document}

\maketitle

\section{Made by duangsuse with love and $ \LaTeXe $}

\begin{center}
Definition for the Y Combinator
\end{center}

\begin{equation}
Y = \lambda f. (\lambda x. f (x x)) (\lambda x. f (x x))
\end{equation}

\begin{center}
Sample Matrix
\end{center}

\begin{equation}
Matrix_0 = \begin{Bmatrix}
	1 & 9 & 3 & 5 & 10 \\
	3 & \mathbf{5} & 9 & 4 & 71 \\
	2 & 3 & 1 & 9 & 34 \\
	9 & 4 & 3 & 2 & 29 \\
	2 & 8 & 4 & 3 & 12
\end{Bmatrix}
\end{equation}

\begin{equation}
{{Matrix_0}_2}_2 \equiv 5
\end{equation}

\begin{equation}
\sum_{i = 2}^4 \sum_{j = 1}^4 {{Matrix_0}_i}_j
\end{equation}

\section{Made by others with unbelievable IQ}

\begin{equation}
x = {-b \pm \sqrt{b^2-4ac} \frac 2a}.
\end{equation}

\begin{equation}
d_i=\displaystyle{\sum_{j=1}^{n} a_{ij}}
\end{equation}

\begin{equation}
\sigma = \sqrt{ \frac{1}{N} \sum_{i=1}^N (x_i -\mu)^2}
\end{equation}

\begin{equation}
(\nabla_X Y)^k = X^i (\nabla_i Y)^k = X^i \left( \frac{\partial Y^k}{\partial x^i} + \Gamma_{im}^k Y^m \right)
\end{equation}

\begin{equation}
\vec{\nabla} \times \vec{F} = \left( \frac{\partial F_z}{\partial y} - \frac{\partial F_y}{\partial z} \right) \mathbf{i}
        + \left( \frac{\partial F_x}{\partial z} - \frac{\partial F_z}{\partial x} \right) \mathbf{j} + \left( \frac{\partial
        F_y}{\partial x} - \frac{\partial F_x}{\partial y} \right) \mathbf{k}
\end{equation}

\begin{equation}
f(x) = \begin{cases} 
	0 & x\leq 0 \\
	\frac{100-x}{100} & 0\leq x\leq 100 \\
	0 & 100\leq x 
\end{cases}
\end{equation}

\begin{equation}
z = \overbrace{
	\underbrace{x}_\text{real} + i
	\underbrace{y}_\text{imaginary}
}^\text{complex number}
\end{equation}

\begin{equation}
C_n^i=\frac{n!}{i!(n-i)!}
\end{equation}
\begin{equation}
B_{i,n}(t)=C_n^i(1-t)^{n-i}t^i
\end{equation}
\begin{equation}
R(t)=\sum_{i=0}^n R_iB_{i,n}(t),\quad 0\leq t\leq 1
\end{equation}

\setmainfont{WenQuanYi Zen Hei}

\section{$\lambda$ - 算子}

早在上世纪 30 年代, 理论上早已出现了许多种不能跑在当下计算机中的 "编程语言", 甚至编程语言这个概念还尚未成熟, 当时人们主要研究 \textit{formal system (形式系统)} 这一领域. 而这其中最为突出的成果为图灵的导师 Alonzo Church 的 $\lambda$-算子, 它是 FP 中非常重要的理论基石, 之后的文章也都全部围绕这一系统进行讨论.

在讨论 $\lambda$ 表达式之前, 我们需要一个 context (上下文), 它类似于我们所说的集合论中的 universe 集合 (全集) $U$, 它保证了我们的论域中拥有独一无二的元素, 写作 $\Gamma$. $\Gamma$ 可以是一个空集 $\emptyset$, 也可以是引入了某元素后的集合, 以此类推. 类比一般的高级编程语言, 它就像符号表, 像一个装有独特变量的相关信息的 哈希表, 甚至也可以是放有变量名的 (数据结构里的) 集合, 当然这些只是比喻.

我们定义一个 $\lambda$ 函数, 在 $\Gamma$ 下引入, 形为

$$\Gamma \vdash \lambda x . x$$

它类似数学中常用的函数 $f(x) = x$, 但这种表示引入了函数名称 $f$, 而 $\lambda$ 函数是 匿名 的, 两者显然并不等同. 这种最基本的 输入即输出 的函数, 被称作 identity function.

和数学中常用的函数表示更加不同的是, 一个 $\lambda$ 函数可以返回另一个 $\lambda$ 函数, 而且对上一个函数的参数是可见的, 即以下函数表示是合法的

$$\lambda x . \lambda y . x$$

简记为

$$\lambda x y . x$$

这一种特性叫做 lexical scoping. 有了这一种特性, 我们便可以做到多参数输入, 单结果输出了. 那具体是怎么做的呢? 首先我们先看一个简单的场景

$$\Gamma \vdash (\lambda x . x) (\lambda y . y)$$

这句话的意思是, 给定一个 context 叫 $\Gamma$, 我们将两个 $\lambda$ 函数添加到 $\Gamma$ 之中. 接着我们用第一个函数的参数 $x$ 代替第二个函数的整体, 按照第一个函数的 "执行规则", 返回一个新的 term (项), 即

$$(\lambda x . x) (\lambda y . y) = _\beta \lambda y . y$$

这一句话的 "执行结果", 即第二个函数本身. 前面将函数加入到 context 中的过程, 我们简称 eval (即 evaluate), 而实际执行这些函数的过程, 简称 apply, 而之前所提到的 "执行规则", 称作 $\beta$-reduction, 它属于一种 term rewriting rules (项的重写规则). 可见, 在 $\lambda$-算子 这个系统中, 最基本的操作即 eval/apply 两个过程的循环.

至于刚刚提到的 多参数输入, 单结果输出, 考虑以下的例子

$$\Gamma \vdash (\lambda xy.x) (\lambda a.a) (\lambda b.b)$$

其 apply 的过程即

$$ (\lambda xy.x) (\lambda a.a) (\lambda b.b) \equiv (\lambda x.\lambda y.x) (\lambda a.a) (\lambda b.b) \\ = _\beta (\lambda y.(\lambda a.a)) (\lambda b.b) \\ = _\beta \lambda a.a $$

这么一个过程. 要注意的是, 在习惯中, 项的 apply 是 left recursive (左递归) 的, 即 $a b c$ 是 $((a b) c)$ 的习惯表达, 而不是 右递归的 $(a (b c))$.

有 $\beta$-reduction 的规则, 是否还存在 $\alpha$-X 这样的规则呢? 这里补充下, 这套系统中还存在有 $\alpha$-conversion 的规则, 在形如

$$\Gamma \vdash (\lambda x . x) (\lambda x . x)$$

的情况下, 这两个项在 $\Gamma$ 中是独一无二的, 但是为了区分二者, 我们可以用如 $y$ 等其他符号替代, 这一种 conversion (转换) 即称为 $\alpha$-conversion, 转换的过程为

$$ (\lambda x . x) (\lambda x . x) \\ = _\alpha (\lambda x . x) (\lambda y . y) $$

\end{document}