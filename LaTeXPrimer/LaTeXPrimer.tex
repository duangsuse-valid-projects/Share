\documentclass[11pt]{article}

\usepackage{amsmath, amssymb}
\usepackage[normalem]{ulem}

% XeTex is utf8 based
%\usepackage[utf8]{inputenc}

\usepackage[T1]{fontenc}
\usepackage{lmodern}
\usepackage[colorlinks]{hyperref}

% Tikz
\usepackage{tikz}

\usepackage{float}
\usepackage{caption}
\usetikzlibrary{shapes.geometric, arrows, positioning}

\tikzstyle{start} = [rectangle, rounded corners, minimum width=3cm, minimum height=1cm,text centered, draw=black]
\tikzstyle{process} = [rectangle, minimum width=3cm, minimum height=1cm, text centered, draw=black]
\tikzstyle{arrow} = [thick,->,>=stealth]

% CJK Support
\usepackage{fontspec, xunicode, xltxtra}

% CJK style
\usepackage[top = 1in, bottom = 1in, left = 1.25in, right = 1.25in]{geometry}

\XeTeXlinebreaklocale "zh"
\XeTeXlinebreakskip = 0pt plus 1pt minus 0.1pt

\renewcommand{\baselinestretch}{1.25}

\parindent 2em

% Set main font
\newfontfamily\opensans{Open Sans}
\newfontfamily\sourcehanserif{Source Han Serif CN}

\usepackage[CJK]{ucharclasses}
\setTransitionsForCJK{\sourcehanserif}{\opensans}{\opensans}

\setmainfont{Open Sans}

\title{$ \LaTeX $ Document Test Page (class aritcle 11pt)}
\author{duangsuse (maketitle, title, author)}
\date{March 10, 2019 with amsmath,amssymb,tikz,hyperref,caption,.$\cdots$}

\begin{document}

\maketitle

\section{Made by duangsuse with love and $ \LaTeXe $}

\begin{equation}
Y = (lambda) \lambda f. (\lambda x. f (x x)) (\lambda x. f (x x))
\end{equation}

\begin{equation}
Matrix_0 = \begin{Bmatrix}
	1 & 9 & 3 & 5 & 10 \\
	3 & \mathbf{5} & 9 & 4 & 71 \\
	2 & 3 & 1 & 9 & 34 \\
	9 & 4 & 3 & 2 & 29 \\
	2 & 8 & 4 & 3 & 12
\end{Bmatrix}
(begin equation, mathbf, Bmatrix)
\end{equation}

\begin{equation}
{{Matrix_0}_2}_2 (equiv) \equiv 5
\end{equation}

\begin{equation}
(sum, dash_sub, accent^sup) \sum_{i = 2}^4 \sum_{j = 1}^4 {{Matrix_0}_i}_j
\end{equation}

\section{Made by others with unbelievable IQ}

\begin{equation}
x = {-b (pm)\pm (sqrt)\sqrt{b^2-4ac} (frac)\frac 2a}.
\end{equation}

\begin{equation}
d_i=(displaystyle)\displaystyle{\sum_{j=1}^{n} a_{ij}}
\end{equation}

\begin{equation}
(sigma)\sigma = \sqrt{ \frac{1}{N} \sum_{i=1}^N (x_i -(mu)\mu)^2}
\end{equation}

\begin{equation}
((nabla)\nabla_X Y)^k = X^i (\nabla_i Y)^k = X^i (left)\left( \frac{(partial)\partial Y^k}{\partial x^i} + (Gamma)\Gamma_{im}^k Y^m (right)\right)
\end{equation}

\begin{equation}
(vec)\vec{\nabla} (times)\times \vec{F} = \left( \frac{\partial F_z}{\partial y} - \frac{\partial F_y}{\partial z} \right) (mathbf)\mathbf{i}
        + \left( \frac{\partial F_x}{\partial z} - \frac{\partial F_z}{\partial x} \right) \mathbf{j} + \left( \frac{\partial
        F_y}{\partial x} - \frac{\partial F_x}{\partial y} \right) \mathbf{k}
\end{equation}

\begin{equation}
f(x) = \begin{cases} (begin cases)
	0 & x(leq)\leq 0 \\
	\frac{100-x}{100} & 0\leq x\leq 100 \\
	0 & 100\leq x 
\end{cases}
\end{equation}

\begin{equation}
z = (overbrace)\overbrace{
	(underbrace)\underbrace{x}_\text{real} + i
	\underbrace{y}_\text{imaginary}
}^\text{complex number}
\end{equation}

\begin{equation}
C_n^i=\frac{n!}{i!(n-i)!}
\end{equation}
\begin{equation}
B_{i,n}(t)=C_n^i(1-t)^{n-i}t^i
\end{equation}
\begin{equation}
R(t)=\sum_{i=0}^n R_iB_{i,n}(t),(quad)\quad 0\leq t\leq 1
\end{equation}

\setmainfont{WenQuanYi Zen Hei}

\section{$\lambda$ - 算子 (section)}

(textit) \textit{formal system italic} \quad (textbf) \textbf{黑体 bold} \\
(textsc) \textsc{Hello 小体大写} \quad (textsf) \textsf{World } \\
\quad (textsl) \textsl{slanted goodbye} \quad (texttt) \texttt{science} \\
(textup) \textup{Sample} \quad (textmd) \textmd{Hello}

\begin{center}
math (begin center)
\end{center}

\begin{math}
(mathbb) \mathbb{Z^+} (\mathbb{N}) \mathbb{Z} \mathbb{Q} \mathbb{R} \mathbb{I} \mathbb{C}
\end{math}

$\lambda$ (lambda) $U$ (U) $\Gamma$ (Gamma) $\emptyset$ (emptyset) (pagebreak) (bigskip)

(Gamma vdash lambda x. x) $\Gamma \vdash \lambda x . x$ \\
$$(forall)\forall x(in)\in\mathbb{Z}_+. x\in{A} \wedge x\ge2$$
$$(beacuse)\because{2=(-1)+3} \quad(therefore)\therefore p$$

\begin{equation}
2n=(-1)+(2n+1),2n+1\in\mathbb{P}
\end{equation}
\begin{equation}
(exists)\exists{n_0}\in\mathbb{N}_+,2n_0(notin)\notin{A}
\end{equation}

\subsection{(subsection) lambda calculus}

(f(x) = x) $f(x) = x$ (f) $f$ (lambda) $\lambda$

$$\Gamma (vdash)\vdash (\lambda x . x) (\lambda y . y)$$
$$(\lambda x . x) (\lambda y . y) = (beta)\beta \lambda y . y$$
$$ (\lambda xy.x) (\lambda a.a) (\lambda b.b) (equiv)\equiv (\lambda x.\lambda y.x) (\lambda a.a) (\lambda b.b) \\ = _\beta (\lambda y.(\lambda a.a)) (\lambda b.b) \\ = _\beta \lambda a.a $$
$$\beta-reduction (alpha)\alpha-X$$
$$ (\lambda x . x) (\lambda x . x) \\ = _\alpha (\lambda x . x) (\lambda y . y) $$

\section{Hello World}

\begin{equation}
\cos(θ+φ)=\cos(θ)\cos(φ)−\sin(θ)\sin(φ)
\end{equation}

\begin{equation}
f(a) = \frac{1}{2(pi)\pi i} (oint)\oint\frac{f(z)}{z-a}dz
\end{equation}

\begin{equation}
(int)\int_D ({\nabla(cdot)\cdot} F)dV=\int_{\partial D} F\cdot ndS
\end{equation}
% From https://liolok.github.io/zh-CN/B%C3%A9zier-%E6%9B%B2%E7%BA%BF%E5%8F%8A%E5%85%B6-deCasteljau-%E5%89%96%E5%88%86%E7%AE%97%E6%B3%95/
\begin{equation}
(\nabla_X Y)^k = X^i (\nabla_i Y)^k = X^i \left( \frac{\partial Y^k}{\partial x^i} + \Gamma_{im}^k Y^m \right)
\end{equation}

一条 $n$ 次 Bézier 曲线可以表示为:$R(t)=\sum_{i=0}^n R_iB_{i,n}(t),\quad 0\leq t\leq 1$

\paragraph{(paragraph) Bezier 曲线}

\label{helo} \begin{quote}
(label helo begin quote) $R_i$ 是控制顶点, 我们可以看出, 一条 $n$ 次 Bézier 曲线有 $n+1$ 个控制顶点, 即 $n$ 次 $n+1$ 阶曲线,$B_{i,n}(t)$ 是 Bernstein 基函数
\end{quote}

\begin{equation}
B_{i,n}(t)=C_n^i(1-t)^{n-i}t^i
\end{equation}
\begin{equation}
C_n^i=\frac{n!}{i!(n-i)!}
\end{equation}

\hypertarget{helo}{(hypertarget helo) hello} (ref helo) \ref{helo}

\begin{itemize}
	\item (begin itemize) (item) 从几何意义上看, 当参数 $t=0$ 时, 对应的是曲线的第 $0$ 个控制顶点;  而当参数 $t=1$ 时, 对应的是曲线的第 $n$ 个控制顶点 。 这就是 Bézier 曲线的端点插值特性, 即 $R(0)=R_0, R(1)=R_1$
	\item 由于二项式系数的对称特性 $C_n^i=C_n^{n-i}$, Bézier 曲线控制顶点的也具有几何地位上的对称性, 即 $\sum_iR_iB_{i,n}(t)=\sum_iR_{n-i}B_{i,n}(t)$
\end{itemize}

\subsection{时间线}
\begin{tabular}{|c|c|c|c|}
\hline
(begin tabular) 全局名称 (and)& 类型 & 格式 & 解释 (newline)(hline)\\
\hline
owner & Integer & Int32 & 时间线所属人 \\
\hline
type & SmallInt & Int16 & 时间线类型 \\
\hline
data & Integer & Int32 & 时间线数据 \\
\hline
created & TimeStamp & Date & 时间线发布( 创建) 时间 (end tabular) \\
\hline
\end{tabular}

\bigskip

% From https://lexuge.github.io/jekyll/update/2018/07/24/lib_blaster%E4%BC%98%E5%8C%96%E7%AC%94%E8%AE%B0.html

\begin{math}
\mu ' = \frac{1s}{3.2731\mu s}\\
(approx)\approx 305530.094714329 pps
\end{math}

\begin{align}
t_{2}\%=\frac{n\cdot{}t_{b}}{n\cdot{}t_{b}}=100\%
\end{align}

\begin{align}(lim limits)\lim\limits_{n(to)\to+(infty)\infty}\frac{t_{b}}{t_{b}+n\cdot{}t_{e}}=0=0\%\end{align}

\begin{align}
\lim
  \limits_{n\to+\infty}k &= \lim\limits_{n\to+\infty} [\frac{t_{b}}{n\cdot (t_{b}+t_{e})}+\frac{n\cdot t_{e}}{n\cdot (t_{b}+t_{e})}]\\
&=\lim\limits_{n\to+\infty}\frac{t_{b}}{n\cdot(t_{b}+t_{e})}+\lim\limits_{n\to+\infty}\frac{n\cdot t_{e}}{n\cdot (t_{b}+t_{e})}\\
&=0+\frac{t_{e}}{t_{b}+t_{e}}\\
&=\frac{t_{e}}{t_{b}+t_{e}}\\
\end{align}

% From https://github.com/Sleepwalking/prometheus-spark/blob/master/writings/ib-ia-hnm/hua-ia-hnm.tex

\newcommand{\matr}[1]{\mathbf{#1}}
(newcommand matr [1] lbrace mathbf sharp1 rbrace)

\begin{equation} \label{linsys}
(matr)\matr{R}x = \matr{b}
\end{equation}

\begin{equation} \label{origh}
(label origh) (hat)\hat{h}(t) = \sum_{k = 1}^{L} a_k(t^i_a)(cos)\cos(2\pi k f_0(t^i_a)(t - t^i_a) + (phi)\phi_k(t^i_a))
\end{equation}

$\matr{b}$ in (\ref{linsys}) is a $(2L + 1) \times 1$ vector with elements $b_k$ as

\begin{equation} \label{rl}
r_{ik} = r_l = \sum_{t = -N}^{N} w_{2N + 1}^2(t) e^{-j 2\pi t f_0 l} (mid)\mid_{l = k - i, -2L \leq l \leq 2L}
\end{equation}

\begin{equation} \label{objective}
\{a_k^*(t^i_a), \phi_k^*(t^i_a)\} = (underset)\underset{a_k(t^i_a), \phi_k(t^i_a)}{(underset of)(arg)\arg(min)\min} \sum_{t = t^i_a - N}^{t^i_a + N} \left(w_{2N + 1}(t)(s(t) - \hat{h}(t))\right)^2
\end{equation}

ref to rl (\ref{rl}) (cite hua-2014) \cite{hua-2014}

(tips: use (input file) to reference external input file to processed by \TeX)
\begin{figure}
\centering

\caption{(begin figure, centering, caption)estimated parameters of harmonics}
\label{fig:harest}
\end{figure}

\section{Greek character (alphabet)}
% https://en.wikipedia.org/wiki/Greek_alphabet

\begin{tabular}{|c|c|c|c|}
\hline
letter & name & IPA & Approximate western European equivalent \\
\hline
A $\alpha$ & alpha & [a] & f(uline)\uline{a}ther \\
\hline
B $\beta$ & beta & [b] & \uline{v}ote \\
\hline
$\Gamma \gamma$ & gamma & [$\gamma$] ~ [j], [ŋ] ~ [jn] & \uline{y}ellow \\
\hline
$\Delta \delta$ & delta & [ð] & \uline{th}en \\
\hline
$Ε(E) \epsilon \varepsilon$ & epsilon & [e] & \uline{p}et \\
\hline
$Z \zeta$ & zeta & [z] & \uline{z}oo \\
\hline
$H \eta$ & eta & [i] & mach\uline{i}ne \\
\hline
$\Theta \theta \vartheta$ & theta & $[\theta]$ & \uline{th}in \\
\hline
$I \iota$ & iota & [i], [ç], [j], [jn] & \uline{i} \\
\hline
$K \kappa$ & kappa & [k] ~ [c] & \uline{k} \\
\hline
$\Lambda \lambda$ & lambda & [l] & \uline{l}antern \\
\hline
$M \mu$ & mu & [m] & \uline{m}usic \\
\hline
$N \nu$ & nu & [n] & \uline{n}et \\
\hline
$\Xi \xi$ & xi & [ks] & fo\uline{x} \\
\hline
$O o$ & omicron & [o] & s\uline{o}ft \\
\hline
$\Pi \pi \varpi$ & pi & [p] & to\uline{p} \\
\hline
$P \rho \varrho$ & rho & [r] & \uline{r} in \\
\hline
$\Sigma \sigma \varsigma$ & sigma & [s] ~ [z] & mu\uline{s}e \\
\hline
$T \tau$ & tau & [t] & coa\uline{t} \\
\hline
$\Upsilon \upsilon$ & upsilon & [i] & \uline{i} \\
\hline
$\Phi \phi$ & phi & [f] & \uline{f}ive \\
\hline
$X \chi$ & chi & [x] ~ [ç] & Scottish lo\uline{ch} \\
\hline
$\Psi \psi$ & psi & [ps] & la\uline{ps}e \\
\hline
$\Omega \omega$ & omega & [o] & s\uline{o}ft \\
\hline

\end{tabular}

\section{Operators}

$(oplus)\oplus (ominus)\ominus (perp)\perp (cap)\cap (cup)\cup (vee)\vee (ni)\ni$ \\
$(sum)\sum (prod)\prod (coprod)\coprod (int)\int (oint)\oint (sqsupset)\sqsupset (subsetneq)\subsetneq (nsubseteq)\nsubseteq (nsupseteq)\nsupseteq$ \\
$(varsupsetneq)\varsupsetneq (supset)\supset (sqsupseteq)\sqsupseteq (star)\star (ast)\ast$ \\
$(rightleftharpoons)\rightleftharpoons (rightarrow)\rightarrow (Leftrightarrow)\Leftrightarrow (circlearrowleft)\circlearrowleft (nRightarrow) \nRightarrow (Rightarrow)\Rightarrow$ \\
$cdots \cdots vdots \vdots ddots \ddots aleph \aleph flat \flat sharp \sharp bigstar \bigstar$ \\
$(complement)\complement (backslash)\backslash (Bbbk)\Bbbk (varnothing)\varnothing (nexists)\nexists (infty)\infty (surd)\surd (top)$ \\
$\top (bot)\bot (neg)\neg (hslash)\hslash (emptyset)\emptyset$

\section{Introduction . Reunderstand PSOLA}

\subsection{PSOLA as a Source-Filter Model}

What leaves me wondering is: seems like there's always a blind spot in all tutorials, slides and papers about PSOLA. In a few sentences they tell you something like,
\footnote{(footnote)A much more detailed yet easy-to-understand video introduction can be found on Professor Simon King's website, \url{(url)http://www.speech.zone/td-psola-the-hard-way/}}

\begin{quotation}
Tikzpicture [node distance = xcm] \\
node (name) [type: start|process(right|below = float| of=name)] {text}; \\
draw [type: arrow] (name) -- (name);
\end{quotation}

\begin{center}
\begin{tikzpicture}[node distance=4cm]
\node (input) [start] {(node)Speech Input};
\node (epoch) [process, right of=input] {Epoch Marking};
\node (window) [process, right of=epoch] {Windowing};
\node (ft) [process, right of=window] {FT};
\node (ift) [process, below = 0.5 of ft] {IFT};
\node (shift-ola) [process, left of=ift] {Shift and Overlap-Add};
\node (output) [start, left of=shift-ola] {Speech Output};
\draw [arrow] (input) -- (epoch);
\draw [arrow] (epoch) -- (window);
\draw [arrow] (window) -- (ft);
\draw [arrow] (ft) -- (ift);
\draw [arrow] (ift) -- (shift-ola);
\draw [arrow] (shift-ola) -- (output);
\end{tikzpicture}
\end{center}

\bigskip

\begin{center}
\begin{tikzpicture}[node distance=4cm]
\node (ptrain) [start] {Pulse Train};
\node (vtract) [process, right of=ptrain] {Vocal-tract Filter};
\draw [arrow] (ptrain) -- (vtract);
\end{tikzpicture}
\end{center}

\begin{center}
\begin{tikzpicture}[node distance=4cm]
\node (ift) [process] {IFT};
\node (shift-ola) [process, right of=ift] {Shift and Overlap-Add};
\node (output) [start, right of=shift-ola] {Speech Output};
\draw [arrow] (ift) -- (shift-ola);
\draw [arrow] (shift-ola) -- (output);
\end{tikzpicture}
\end{center}

\begin{thebibliography}{99}

\bibitem{hua-2014}{(begin thebibliography 99 bibitem hua-2014) Hua, Kanru. ``A method to improve the extraction quality of periodic component of speech". Patent Application. CN201410457379. 2014.}
\end{thebibliography}

\end{document}
