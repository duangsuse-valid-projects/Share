% 选主元的高斯-约当(Gauss-Jordan)消元法解线性方程组/求逆矩阵
% author: learnhard
% origin: http://www.codelast.com/?p=1288

\documentclass[a4paper, 8pt]{article}

%%
\usepackage[utf8]{inputenc}
\usepackage[T1]{fontenc}

\usepackage{lmodern}

\usepackage[top = 1em, bottom = 1em, left=4em, right=4em]{geometry}

\usepackage[normalem]{ulem}
\usepackage[colorlinks]{hyperref}

\usepackage{amsmath, amssymb}

% CJK support
\usepackage{fontspec, xunicode, xltxtra}
\XeTeXlinebreaklocale "zh"
\XeTeXlinebreakskip 0pt plus 1pt minus 0.1pt

%%
\newfontfamily\opensans{Open Sans}
\newfontfamily\sourcehanserif{Source Han Serif CN}

\usepackage[CJK]{ucharclasses}
\setTransitionsForCJK{\sourcehanserif}{\opensans}{\opensans}

\setmainfont{Open Sans}

%%
\newcommand{\matr}[1]{\mathbf{#1}}
\newcommand*\rfrac[2]{{}^{#1}\!/_{#2}}

%%
\parindent 2em
\renewcommand{\baselinestretch}{1.25}

\title{选主元的高斯-约当 (Gauss-Jordan) 消元法解线性方程组 / 求逆矩阵}
\author{learnhard (codelast.com)}
\date{2011 年 03 月 13 日}

%%
\begin{document}
\maketitle
\begin{abstract}
Tagged on:	Gauss-Jordan	optimization	最优化	消元	线性方程组	选主元	高斯-约当 \\
Category:	Algorithm, 原创. 14 Comments 
\end{abstract}

	选主元的高斯-约当 (Gauss-Jordan) 消元法在很多地方都会用到, 例如求一个矩阵的逆矩阵/解线性方程组 ( 插一句: \href{http://www.codelast.com/?p=29}{LM 算法}求解的一个步骤), 等等 。 它的速度不是最快的, 但是它非常稳定\footnote{来自网上的定义: 一个计算方法, 如果在使用此方法的计算过程中, 舍入误差得到控制, 对计算结果影响较小, 称此方法为数值稳定的}, 同时它的求解过程也比较清晰明了, 因而人们使用较多 。 下面我就用一个例子来告诉你 Gauss-Jordan 法的求解过程吧 。 顺便再提及一些注意事项以及扩展话题 。

对本文中所提到的“主元”等概念的解释, 可以参考\href{http://www.codelast.com/?page_id=963}{此链接} 。 \\
假设有如下的方程组:

\begin{equation}
\left\{
\begin{array}{lr}
x_1 + 3x_2 + x_3 = 11 \\
2x_1 + x_2 + x_3 = 8 \\
2x_1 + 2x_2 + x_3 = 10
\end{array}
\right.
\end{equation}

写成矩阵形式就是:$\matr{A}\matr{X}=\matr{B}$, 其中:

\begin{equation}
A = \begin{pmatrix}
1 & 3 & 1 \\
2 & 1 & 1 \\
2 & 2 & 1 \\
\end{pmatrix}, B = \begin{pmatrix}
11 \\ 8 \\ 10
\end{pmatrix}
\end{equation}

且 $X=(X_1, X_2, X_3)^T$ 。



现对矩阵 $\matr{A}$ 作\href{http://www.codelast.com/?page_id=963}{初等行变换}, 同时矩阵 $\matr{B}$ 也作同样的初等变换, 则当 $\matr{A}$ 化为单位矩阵的时候, 有:

\begin{equation}
\begin{pmatrix}
1&0&0\\
0&1&0\\
0&0&1
\end{pmatrix} \matr{X} = \begin{pmatrix}
1\\2\\4
\end{pmatrix}
\end{equation}

显而易见, 我们得到了方程组的解 $\matr{X}=(1, 2, 4)^T$。

	所以, 我们要以一定的策略, 对 $\matr{A}$ 和 $\matr{B}$ 施以一系列的初等变换\footnote{\label{elementary_trans} 过程如下: \begin{itemize}\item[1] 交换矩阵的两行或列
\item[2] 用一个不为零的数乘矩阵的某一行或列
\item[3] 用一个数乘矩阵某一行或列加到另一行或列上\end{itemize}}, 当 $\matr{A}$ 化为单位矩阵的时候,$\matr{B}$ 就为方程组的解 。

	选主元的 G-J 消元法通过这样的方法来进行初等变换: 在每一个循环过程中, 先寻找到主元, 并将主元通过行变换 ( 无需列变换) 移动到矩阵的主对角线上, 然后将主元所在的行内的所有元素除以主元, 使得主元化为 $1$; 然后观察主元所在的列上的其他元素, 将它们所在的行减去主元所在的行乘以一定的倍数, 使得主元所在的列内 、 除主元外的其他元素化为 $0$, 这样就使得主元所在的列化为了单位矩阵的形式 。 这就是一个循环内做的工作 。 然后, 在第二轮循环的过程中, 不考虑上一轮计算过程中主元所在的行和列内的元素, 在剩下的矩阵范围内寻找主元, 然后( 如果其不在主对角线上的话) 将其移动到主对角线上, 并再次进行列的处理, 将列化为单位矩阵的形式 。 余下的步骤依此类推 。 具体的计算过程的一个例子, 请看下面我举的求逆矩阵的过程 。

	如果要解系数矩阵相同 、 右端向量不同的 $n$ 个方程组, 在设计程序的时候, 没有必要“解 $n$ 次方程组”, 我们完全可以在程序中, 将所有的右端向量以矩阵的数据结构( 类似于二维数组) 来表示, 在系数矩阵作行变换的时候, 矩阵里的每一个右端向量也做同样的变换, 这样, 我们在一次求解运算的过程中, 实际上就是同时在解 $n$ 个方程组了, 这是要注意的地方 。 \\


%%

	那么,G-J 法为什么可以用来求逆矩阵?
	
	假设 \label{ax_eq_e}$\matr{A}\matr{X}=\matr{E}$, 其中,$\matr{A}$ 为 $n$ 阶系数矩阵( 与上面的解线性方程组对照);$\matr{E}$ 为单位矩阵, 即 $\matr{E}=(e_1,e_2,\cdots,e_n)$, 其中 $e_i (i=1,2,\cdots,n)$ 为单位列向量;$\matr{X}$ 为 $n$ 个列向量构成的矩阵, 即$\matr{X}=(x_1,x_2,\cdots,x_n)$, 其中 $x_i (i=1,2,\cdots,n)$ 为列向量 。

	于是, 可以把等式 $\hyperref[ax_eq_e]{\matr{AX=E}}$ 看成是求解 $n$ 个线性方程组 $Ax_i=e_i (i=1,2,\cdots,n)$, 求出了所有的 $x_i$ 之后, 也即得到了矩阵 $\matr{X}$。而由 $\hyperref[ax_eq_e]{\matr{AX=E}}$ 可知, 矩阵 $\matr{X}$ 是 $\matr{A}$ 的逆矩阵, 即 $\matr{X}=\matr{A}-1$。这样, 就求出了 $\matr{A}$ 的逆矩阵了 。于是, 求逆矩阵的过程被化成了解线性方程组的过程, 因此我们可以用 Gauss-Jordan 消元法来求逆矩阵 。

	求逆矩阵时, 系数矩阵 $\matr{A}$ 和单位矩阵 $\matr{E}$ 可以共用一块存储区, 在每一次约化过程中, 系数矩阵逐渐被其逆矩阵替代 。 \\

	在这里, 我用一个实际的例子来说明 G-J 法求逆矩阵的过程:

有如下的方程组:

\begin{equation}
\left\{\begin{array}{lr}
x_1 + 3x_2 + x_3 = 11 \\
2x_1 + x_2 + x_3 = 8 \\
2x_1 + 2x_2 + x_3 = 10
\end{array}\right.
\end{equation}

显而易见, 该方程组对应的系数矩阵 $\matr{A}$ 和右端向量矩阵 $\matr{B}$( 此处只有一个右端向量) 分别为:

\begin{equation}
A = \begin{pmatrix}
1 & 3 & 1 \\
2 & 1 & 1 \\
2 & 2 & 1 \\
\end{pmatrix}, B = \begin{pmatrix}
11 \\ 8 \\ 10
\end{pmatrix}
\end{equation}

其实在求逆矩阵的过程中, 矩阵 $\matr{B}$ 无关紧要, 可以忽略, 不过此处还是把它写出来了 。下面, 把单位矩阵 $\matr{E}$ 附在 $\matr{A}$ 的右边, 构成另一个矩阵 $(\matr{A}|\matr{E})$:

\begin{equation}
(\matr{A}|\matr{E}) = \begin{pmatrix}
1 & 3 & 1 & \vline & 1&0&0 \\
2 & 1 & 1 & \vline & 0&1&0 \\
2 & 2 & 1 & \vline & 0&0&1\\
\end{pmatrix}
\end{equation}

下面, 我们就通过矩阵的初等变换 $\hyperref[elementary_trans]{elementary transformation}$, 将 $\matr{A}$ 化为单位矩阵 $\matr{E}$, 而 $\matr{E}$ 则化为了 $\matr{A}$ 的逆矩阵 。以下是转化步骤:

\begin{itemize}
\item[1] 主元选为 $3$, 所以将 $Row1$ ( 第一行) 与 $Row2$ ( 第二行) 交换:

\begin{equation}
\begin{pmatrix}
2 & 1 & 1 & \vline & 0&1&0 \\
1 & 3 & 1 & \vline & 1&0&0 \\
2 & 2 & 1 & \vline & 0&0&1\\
\end{pmatrix}
\end{equation}

\item[2] 主元所在行的所有元素除以主元:

\begin{equation}
\begin{pmatrix}
2 & 1 & 1 & \vline & 0&1&0 \\
\rfrac{1}{3} & \rfrac{3}{3} & \rfrac{1}{3} & \vline & \rfrac{1}{3}&0&0 \\
2 & 2 & 1 & \vline & 0&0&1\\
\end{pmatrix}
\end{equation}

\item[3] $Row1 - Row2$ , $Row3 - (2 \times Row2)$ :

\begin{equation} \label{row1_sub_row2_andon}
\begin{pmatrix}
2-\rfrac{1}{3} & 1-\rfrac{3}{3} & 1-\rfrac{1}{3} & \vline & 0-\rfrac{1}{3}&1&0 \\
\rfrac{1}{3} & \rfrac{3}{3} & \rfrac{1}{3} & \vline & \rfrac{1}{3}&0&0 \\
2-\rfrac{1\times 2}{3} & 2-\rfrac{3\times 2}{3} & 1-\rfrac{1\times 2}{3} & \vline & 0-\rfrac{1\times 2}{3}&0&1\\
\end{pmatrix}
\end{equation}

\ref{row1_sub_row2_andon} 式最终被计算为:

\begin{equation}
\begin{pmatrix}
1\frac{2}{3} & 0 & \rfrac{2}{3} & \vline & -\rfrac{1}{3} &1&0 \\
\rfrac{1}{3} & 1 & \rfrac{1}{3} & \vline & \rfrac{1}{3} &0&0 \\
1\frac{1}{3} & 0 & \rfrac{1}{3} & \vline & -\rfrac{2}{3} &0&1
\end{pmatrix}
\end{equation}

( 现在, 原来的矩阵 $\matr{A}$ 有一列被化为了单位阵的形式)

\item[4] 重新选主元, 这一次主元选为 $5/3$ , 于是 $Row1 \div \rfrac{5}{3}$ ( 主元所在行的所有元素除以主元):

\begin{equation} \label{row1_div_frac_5_3}
\begin{pmatrix}
\rfrac{5}{3} \div \rfrac{5}{3} & 0 \div \rfrac{5}{3} & \rfrac{2}{3} \div \rfrac{5}{3} & \vline & -\rfrac{1}{3} \div \rfrac{5}{3} &1 \div \rfrac{5}{3} & 0 \div \rfrac{5}{3} \\
\rfrac{1}{3} & 1 & \rfrac{1}{3} & \vline & \rfrac{1}{3} &0&0 \\
1\frac{1}{3} & 0 & \rfrac{1}{3} & \vline & -\rfrac{2}{3} &0&1
\end{pmatrix}
\end{equation}

\ref{row1_div_frac_5_3} 式最终被计算为:

\begin{equation} \label{row1_div_frac_5_3_result}
\begin{pmatrix}
1 & 0 & \rfrac{2}{5} & \vline & -\rfrac{1}{5} & \rfrac{5}{3} & 0 \\
\rfrac{1}{3} & 1 & \rfrac{1}{3} & \vline & \rfrac{1}{3} &0&0 \\
1\frac{1}{3} & 0 & \rfrac{1}{3} & \vline & -\rfrac{2}{3} &0&1
\end{pmatrix}
\end{equation}

\item[5] $Row2 - (\rfrac{1}{3} \times Row1)$ , $Row3 - (\rfrac{4}{3} \times Row1)$ :

\begin{equation} \label{row2_sub_frac_1_3mul_row1_andon}
\begin{pmatrix}
1 & 0 & \rfrac{2}{5} & \vline & -\rfrac{1}{5} & \rfrac{5}{3} & 0 \\
\rfrac{1}{3}-(1\times\rfrac{1}{3}) & 1-(0\times\rfrac{1}{3}) & \rfrac{1}{3}-(\rfrac{2}{5}\times\rfrac{1}{3}) & \vline & \rfrac{1}{3}-(-\rfrac{1}{5}\times\rfrac{1}{3}) &0-(\rfrac{5}{3}\times\rfrac{1}{3})&0-(0\times\rfrac{1}{3}) \\
1\frac{1}{3} & 0 & \rfrac{1}{3} & \vline & -\rfrac{2}{3} &0&1
\end{pmatrix}
\end{equation}

\begin{equation} \label{row2_sub_frac_1_3mul_row1_andon_step2}
\begin{pmatrix}
1 & 0 & \rfrac{2}{5} & \vline & -\rfrac{1}{5} & \rfrac{5}{3} & 0 \\
\rfrac{1}{3}-(1\times\rfrac{1}{3}) & 1-(0\times\rfrac{1}{3}) & \rfrac{1}{3}-(\rfrac{2}{5}\times\rfrac{1}{3}) & \vline & \rfrac{1}{3}-(-\rfrac{1}{5}\times\rfrac{1}{3}) &0-(\rfrac{5}{3}\times\rfrac{1}{3})&0-(0\times\rfrac{1}{3}) \\
\rfrac{4}{3}-(1\times\rfrac{4}{3}) & 0-(0\times\rfrac{4}{3}) & \rfrac{1}{3}-(\rfrac{2}{5}\times\rfrac{4}{3}) & \vline & -\rfrac{2}{3}-(-\rfrac{1}{5}\times\rfrac{4}{3}) & 0-(\rfrac{5}{3}\times\rfrac{4}{3}) & 1-(0\times\rfrac{4}{3})
\end{pmatrix}
\end{equation}

式子 \ref{row2_sub_frac_1_3mul_row1_andon_step2} 可以被简化为:

\begin{equation} \label{row2_sub_frac_1_3mul_row1_andon_result}
\begin{pmatrix}
1&0&\frac{2}{5} &\vline& -\frac{1}{5} & \frac{3}{5} & 0 \\
0&1&\frac{1}{5} &\vline& \frac{2}{5} & -\frac{1}{5} & 0  \\
0&0&-\frac{1}{5} &\vline& -\frac{2}{5} & -\frac{4}{5} & 1
\end{pmatrix}
\end{equation}


( 现在, 原来的矩阵 $\matr{A}$ 又有一列被化为了单位阵的形式)

\item[6] 重新选主元, 这一次主元选为 $-1/5$ , 于是 $Row3 \div (-1/5)$ ( 主元所在行的所有元素除以主元):

\begin{equation}
\begin{pmatrix}
1&0&\frac{2}{5} &\vline& -\frac{1}{5} & \frac{3}{5} & 0 \\
0&1&\frac{1}{5} &\vline& \frac{2}{5} & -\frac{1}{5} & 0  \\
0&0&1 &\vline& 2 & 4 & -5
\end{pmatrix}
\end{equation}

\item[7] $Row1 - (2/5) \times Row3$ , $Row2 - (1/5) \times Row3$ :

\begin{equation}
\begin{pmatrix}
1&0&0 &\vline& -1 & -1 & 2 \\
0&1&0 &\vline& 0 & -1 & 1  \\
0&0&1 &\vline& 2 & 4 & -5
\end{pmatrix}
\end{equation}

现在, 原来的矩阵 $\matr{A}$ 的所有列都被化为了单位阵的形式 。

\end{itemize}

可见, 以上过程非常适合于计算机编程求解 。

	至此, 我们完成了从 $\matr{A}$ 到 $\matr{E}$ 的转换, 这个过程中使用了选主元的方法, 但没有使用列交换 。于是, 原来的单位矩阵 $\matr{E}$ 就变成了 $\matr{A}-1$, 即:

\begin{equation}
A^{-1} = \begin{pmatrix}
-1 & -1 & 2 \\
0 & -1 & 1 \\
2 & 4 & 5
\end{pmatrix}
\end{equation}

	有人说, 在进行转化的过程中, 如果某一步发现选中的主元为 $0$, 怎么办? 当然, 这种情况就进行不下去了 ( 矩阵是奇异的) 。

\end{document}


