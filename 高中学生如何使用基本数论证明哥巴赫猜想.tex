\documentclass[11pt]{article}
\usepackage{amsmath, amssymb}

% XeTex is utf8 based
%\usepackage[utf8]{inputenc}

\usepackage[T1]{fontenc}
\usepackage{lmodern}

% CJK Support
\usepackage{fontspec, xunicode, xltxtra}

% CJK style
\usepackage[top = 1in, bottom = 1in, left = 1.25in, right = 1.25in]{geometry}

\XeTeXlinebreaklocale "zh"
\XeTeXlinebreakskip = 0pt plus 1pt minus 0.1pt

\renewcommand{\baselinestretch}{1.25}

\parindent 2em

% Set main font

\newfontfamily\opensans{Open Sans}
\newfontfamily\sourcehanserif{Source Han Serif CN}

\usepackage[CJK]{ucharclasses}
\setTransitionsForCJK{\sourcehanserif}{\opensans}{\opensans}

\setmainfont{Open Sans}

\title{哥巴赫猜想的证明}
\author{知乎用户 $\textbf{@ 证明}$}

\begin{document}

\maketitle

\begin{center}
	思路: 有借有还, 再借不难; 分类讨论, 逐一判断
\end{center}

\section{准备}

\begin{itemize}
	\item 设集合 $\textit{A}$ 为所有满足两个质数之和的偶数的集合, 且此时质数包括正质数和负质数
	\item 设集合 $\textit{B}$ 为所有满足两个质数之和的偶数的集合, 且此时质数只包括正质数
	\item $\forall$ 任意大于等于 $4$ 的偶数 \\ 均可以表示为 $(6k-2), 6k, (6k + 2)$ 中的一种, 其中 $k\in\mathbb{N}_+$
	\item 约定全体素数集为 $\mathbb{P}$, 且有 $k\in\mathbb{N}_+$
\end{itemize}

目的: 证明 $\forall{n}\in\mathbb{N}_+$ 且 $n\ge2$ 则 $2n\in{B}$

\section{证明}

\paragraph{1}

$\because{2=(-1)+3}$ \\
$\therefore$ 得到一个新猜想, 即

\begin{equation}
	\forall{n}\in\mathbb{N}_+,2n\in{B}
\end{equation}

或

\begin{equation}
	2n=(-1)+(2n+1),2n+1\in\mathbb{P}
\end{equation}

将其称为哥德巴赫猜想变式一

\pagebreak

如果把 $-1$ 归到质数集中, 可得命题:

\begin{quotation}
	若 $-1$ 是质数, 则 $\forall{n}\in\mathbb{N}_+,2n\in{A}$ , 且 $-1$ 为唯一的负质数
\end{quotation}

这个命题的逆否命题为:

\begin{quotation}
	若 $\exists{n_0}\in\mathbb{N}_+,2n_0\notin{A}$ , 则 $-1$ 不是质数
\end{quotation}

当 $n_0=1$ 时, 若 $2\notin{A}$ , 则 $-1$ 不是质数: 这是一个真命题, 所以该存在命题是真命题

由于原命题和其逆否命题具有等价关系, 所以原命题是真命题, 哥德巴赫猜想变式一正确

令 $2n=6k+2$ , 则有 $6k+2\in{B}$ 或 $6k+2=-1+(6k+3), (6k+3)\in\mathbb{P}$ 成立 。 显然 $6k+3=3(2k+1)\notin\mathbb{P}$ , 从而有 $6k+2\in{B}$

\paragraph{2}

$\because{2=(-3)+5}$, 又可以提出另一个猜想:

\begin{equation}
	\forall{n}\in\mathbb{N}_+,2n\in{B}
\end{equation}

或

\begin{equation}
	2n=-3+(2n+3),2n+3\in\mathbb{P}
\end{equation}

成立, 将其称为哥德巴赫猜想变式二

如果把 $-3$ 归到质数的集合中, 可得命题:

\begin{quotation}
	若 $-3$ 是质数, 则 $\forall{n}\in\mathbb{N}_+,2n\in{A}$ , 且 $-3$ 为唯一的负质数
\end{quotation}

这个命题的逆否命题为:

\begin{quotation}
	若 $\exists{n_0}\in\mathbb{N}_+,2n_0\notin{A}$ , 则 $-3$ 不是质数
\end{quotation}


当 $n_0=1$ 时, 若 $2\notin{A}$ , 则 $-3$ 不是质数: 这是一个真命题, 所以该存在命题是真命题

由于原命题和其逆否命题具有等价关系, 所以原命题是真命题, 哥德巴赫猜想变式二正确

令 $2n=6k$ , 则有 $6k\in{B}$ 或 $6k=-3+(6k+3)$,$(6k+3)\in\mathbb{P}$ 成立 。 显然 $6k+3=3(2k+1)\notin\mathbb{P}$ , 从而有 $6k\in{B}$ 。

\paragraph{3}

同理可证 $\forall{n}\in\mathbb{N}_+$,$2n\in{B}$ 或 $2n=-5+(2n+5)$,$2n+5\in\mathbb{P}$ 成立 。 取 $2n=6k-2 $, 同理可得 $6k-2\in{B}$

%(编者注:mmp 实在写不下去了,就是把上面 -1 和 -3 换成 -5 而已)


\paragraph{4}
由于 $6k-2,6k,6k+2\in{B}$ 且 $k\in\mathbb{N}_+$ , 故 $\forall{n}\in\mathbb{N}_+$ 且 $n\ge2$ , $2n\in{B}$ , 从而哥德巴赫猜想正确

\end{document}